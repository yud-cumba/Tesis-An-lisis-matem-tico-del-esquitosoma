\section{Resolución de nuestra EDP}
Nuestro objetivo es resolver la siguiente EDP. Dado $s>0,$
$$\begin{cases}
    G_t+\tilde{\mu}_1(z-1)G_z=\frac{1}{2}\nu_1Y(t)(z-1)G\\G(s,z)=z^m
\end{cases}$$
En este caso nuestra curva inicial parametrizada por $r$ está dada por $\gamma(r)=(s,r)$ con condición inicial $\phi(r)=r^m$.\\
Nuestro sistema de ecuaciones diferenciales para las curvas integrales $\{t_r:t_r(u)\}_{r\in\R}$, $\{z_r:z_r(u)\}_{r\in\R}$ y $\{v_r:v_r(u)\}_{r\in\R}$ toma la forma
\begin{eqnarray}
    \begin{cases}
    \label{edp-curvas-caracteristicas}
        t_r'=1, & t_r(0)=s\\
        z_r'=\mu_1(z_r-1), & z_r(0)=r\\
        v'_r(u)=\frac{1}{2}\nu_1Y(t_r)(z_r-1)v, & v_r(0)=r^m,
    \end{cases}
\end{eqnarray}
Resolviendo este sistema para cada $r\in\R$ se obtiene que explícitamente las curvas integrales $\{t_r:t_r(u)\}_{r\in\R}$ y $\{z_r:z_r(u)\}_{r\in\R}$ están dadas por $t_r(u)=u+s$ y $z_r(u)=(r-1)e^{\mu_1 u}+1$.\\
Reemplazamos estas curvas en la ecuación \ref{edp-curvas-caracteristicas}
$$v'_r-\frac{1}{2}\nu_1Y(u+s)(r-1)e^{\mu_1 u}v= 0, \quad v_r(0)=r^m$$
Resolvamos esta EDO lineal por el método del factor integrante, el cual en  este caso sería
$$\exp\bigg(-\int_0^u \frac{1}{2}\nu_1Y(x+s)(r-1)e^{\mu_1 u}\bigg).$$
Si denotamos
$$U(u)=-\frac{1}{2}\nu_1(r-1)\int_0^u Y(x+s)e^{\mu_1x} dx $$
haciendo el cambio de variable $x=x-s$
$$U(u)=-\frac{1}{2}\nu_1(r-1)\int_{s}^{u+s}Y(x)e^{u(x-s)}dx
=-\frac{1}{2}\nu_1(r-1)\bigg[\int_0^{u+s}Y(x)e^{\mu_1 (x-s)}dx-\int_0^{s}Y(x)e^{\mu_1 (x-s)}dx\bigg]$$
$$=-\frac{1}{2}\nu_1(r-1)e^{-\mu_1 s}\bigg[\int_0^{u+s}Y(x)e^{\mu_1 x}dx-\int_0^{t_0}Y(x)e^{\mu_1 x}dx\bigg]$$
$$=-\frac{1}{2}\nu_1(r-1)e^{-\mu_1 s}(e^{\mu_1 u}e^{-\mu_1 u})\bigg[\int_0^{u+s}Y(x)e^{\mu_1 x}dx-\int_0^{s}Y(x)e^{\mu_1 x}dx\bigg]$$
$$=-\frac{1}{2}\nu_1(r-1)\bigg[e^{\mu_1 u}\bigg(e^{-\mu_1 (s+u)}\int_0^{u+s}Y(x)e^{\mu_1 x}dx\bigg)-\bigg(e^{-\mu_1 s}\int_0^{s}Y(x)e^{\mu_1 x}dx\bigg)\bigg].$$
Si denotamos $$\beta(s)=e^{-\mu_1 s}\int_0^sY(x)e^{\mu_1 x}dx$$ tenemos 
$$U(u)=-\frac{1}{2}\nu_1(r-1)[e^{\tilde{\mu_1}u}\beta(u+s)-\beta(s)]$$
Gracias al método del factor integrante la solución general de $$v_r(u)=Ce^{U(u)}$$
donde $$v_r(0)=r^m$$
entonces
$$C \exp\bigg(\frac{1}{2}\nu_1(r-1)[e^{\tilde{\mu_1}u}\beta(0+s)-\beta(s)]\bigg)=C=r^m$$
Por lo tanto,
$$v_r(u)=r^m\exp\bigg(\frac{1}{2}\nu_1(r-1)[e^{\tilde{\mu_1}u}\beta(u+s)-\beta(s)]\bigg)$$
La función inversas de las curvas $t_r$ y $s_r$ vendrían a ser
$$u=t-s\quad,r=(z-1)e^{-\mu_1(t-s)}+1$$
Reemplazando estas curvas en la curva $v_r(u)$ se tiene que
$$G(t,z)=[e^{-\mu_1(t-s)}(z-1)+1]^m\exp\bigg(\frac{1}{2}\nu_1((z-1)e^{-\mu_1(t-s)})[e^{\mu_1(t-s)}\beta(t)-\beta(s)]\bigg)$$
La expresión 
\begin{eqnarray}
    G(t,z)=[e^{-\mu_1(t-s)}(z-1)+1]^m \exp\bigg(\frac{1}{2}\nu_1(z-1)\big(\beta(t)-e^{-\mu_1(t-s)}\beta(s)\big)\bigg)\label{gmk}
\end{eqnarray}
Recordemos que esta función generadora $G_t$ está relacionada a la variable aleatoria $M_k(t)$ y con la probabilidad de transición $P_{m,n}$.
Por ello, por un momento usemos la notación que indica la variable aleatoria de la cual fue generada la función $G_t$, es decir, $G_{M_k(t)}(z; s,m)=\sum_{n=1}^\infty z^n P_{m,n}(s,t)$.\\ $X\sim Bin\big(m,e^{-\tilde{\mu_1}(t-s)}\big)$ , $Y\sim Pois\big(\frac{1}{2}\nu_1(\beta(t)-\beta(s)e^{-\tilde{\mu_1}(t-s)})\big)$.
Es decir $$f_X(x)={m \choose x}p^x(1-p)^{m-x}$$
$$f_Y(y)=e^{-\lambda}\frac{\lambda^y}{y!}$$ donde $\lambda=\frac{1}{2}\nu_1(\beta(t)-\beta(s)e^{-\tilde{\mu_1}(t-t_0)})$ , $p=e^{-\tilde{\mu_1}(t-s)}$.\\Para estas variables aleatorias tenemos que
$$G_{X(t)}(z)=[e^{-u(t-s)}(z-1)+1]^m$$
$$G_{Y(t)}(z)=exp[\frac{1}{2}\nu_1(z-1)(\beta(t)-e^{-u(t-s)}\beta(z))]$$
Por la proposición \ref{gmultiplica} y por \ref{gmk} $$G_{X(t)+Y(t)}(z)=G_{X(t)}G_{Y(t)}(z)=G_{M_K(t)}(z;s,m)$$
Entonces $X(t)+Y(t)$ y $  M_k(t)$ tienen la misma distribución. Esto es
$$\mathbf{P}_{m,n}(s,t)=P(M_k(t)=n|M_k(s)=m)=P(X(t)+Y(t)=n|M_k(s)=m)=P(X(t)+Y(t)=n)f_{X(t)+Y(t)}(n)$$
El cual es una convolución de las variables aleatorias $X(t)$ e $Y(t)$ definida en \ref{convolucion}. Entonces $$P_n(t)=(f_{X(t)}*f_{Y(t)})(n)=\sum_{j=0}^n f_{X(t)}(j)f_{Y(t)}(n-j)=\sum_{j=0}^m {m \choose j}p^j(1-p)^{m-j} e^{-\lambda}\frac{\lambda^{n-j}}{(n-j)!}$$
Explícitamente, se tiene que $$P_n(t)=\exp\bigg(-\frac{1}{2}\nu_1(\beta(t)-\beta(t_0)e^{-\tilde{\mu_1}(t-t_0)})\bigg)\sum_{j=0}^n{m \choose j}(e^{-\tilde{\mu_1}(t-t_0)})^j(1-e^{-\tilde{\mu_1}(t-t_0)})^{m-j}\frac{\big(\frac{1}{2}\nu_1(\beta(t)-\beta(t_0)e^{-\tilde{\mu_1}(t-t_0)})\big)^{n-j}}{(n-j)!}$$
Además 
\begin{Lem}
$$E(X(t)|M_k(0)=j)=je^{-\tilde{\mu}_1 t}$$
    \begin{proof}
        Como en este caso se conoce el valor de la distribución inicial $M_k(0)=j$
        $$f_{(X(t)|M_k(0))}(x|j)=
        {j \choose x} p_0^x(1-p_0)^{j-x} \quad x=0,1,\ldots,j$$
        Donde $p_0=e^{-\tilde{\mu}_1 t}$
        $$E(X(t)|M_k(0)=j)=\sum_{x=0}^m x f_{(X(t)|M_k(0))}(x|j)=\sum_{x=1}^j x{j \choose x}p_0^x(1-p_0)^{j-x}=\sum_{x=1}^j \frac{j!}{(x-1)!(j-x)!}p_0^x(1-p_0)^{j-x}$$ 
        $$=jp_0\sum_{x=1}^j \frac{(m-1)!}{(x-1)!(m-x)!}p_0^{x-1}(1-p_0)^{m-x}=mp_0\sum_{x=1}^m {j-1\choose x-1} p_0^{x-1}(1-p_0)^{j-x}$$ $$=jp\sum_{x=0}^{j-1} {j-1\choose x} p_0^{x}(1-p_0)^{(j-1)-x}=jp$$
        Como $p_0=e^{-\tilde{\mu}_1 t}$ se tiene el resultado .
    \end{proof}
\end{Lem}
\begin{Lem}
    $$E(Y(t)|M_k(0)=j)=\frac{1}{2}\nu_1\beta(t)$$
    \begin{proof}
        $$f_{(Y(t)|M_k(0))}(y|j)=e^{-\gamma_0}\frac{\gamma_0^y}{y!}$$ 
        Donde $$\gamma_0=\frac{1}{2}\nu_1\beta(t)$$
        $$E(Y(t)|M_k(0)=j)=\sum_{y=1}^\infty yf_{(Y(t)|M_k(0))}(y|j)=\sum_{y=1}^\infty ye^{-\gamma_0}\frac{\gamma_0^y}{y!}$$
        $$=\gamma_0 e^{-\gamma_0}\sum_{y=1}^\infty  \frac{\gamma_0^{y-1}}{(y-1)!}=\gamma_0 e^{-\gamma_0}e^{\gamma_0}=\gamma_0=\frac{1}{2}\nu_1\beta(t)$$
    \end{proof}
\end{Lem}
$$E(M_k(t))=\sum_{n=1}^\infty nP_n(t)$$
\begin{eqnarray}
    P_n(t)=\sum_{j=0}^\infty P_{j,n}(t)P_j(0)
    \label{particion_pn}
\end{eqnarray}
Entonces $$E(M_k(t))=\sum_{n=1}^\infty n \sum_{j=1}^\infty P_{j,n}(0,t)P_k(0)=\sum_{j=1}^\infty P_j(0)\sum_{n=1}^\infty n P_{j,n}(0,t)=\sum_{j=1}^\infty q^{(k)}_jE(M_k(t)|M_k(0)=j)$$ $$=\sum_{j=1}^\infty q^{(k)}_jE(X(t)+Y(t)|M_k(0)=j)=\sum_{j=1}^\infty q^{(k)}_j[E(X(t)|M_k(0)=j)+E(Y(t)|M_k(0)=j)]$$ $$=\sum_{j=1}^\infty q^{(k)}_j\big(je^{-\tilde{\mu}_1t}+\frac{1}{2}\beta(t)\big)$$Analogamente se tiene que $$E(F_k(t))=\sum_{j=1}^\infty p^{(k)}_j\big(je^{-\tilde{\mu}_1t}+\frac{1}{2}\nu_1\beta(t)\big)$$
Entonces $$E(W_k(t))=E(M_k(t))+E(F_k(t))=\sum_{j=1}^\infty( p^{(k)}_j+q^{(k)}_j)\big(je^{-\tilde{\mu}_1t}+\frac{1}{2}\nu_1\beta(t)\big)$$
Un interés particular es el caso cuando $M_k(0)$ y $F_k(0)$ tiene una distribución de Poisson con el mismo parámetro $\frac{1}{2}\omega_k$, de \ref{particion_pn}
$$P(M_k(t)=n)=P(F_k(t)=n)=\sum_{m=0}^\infty P_{m,n}(t) \frac{(\frac{1}{2}\omega_k)^m e^{-\frac{1}{2}\omega_k}}{m!}$$

$$=\sum_{m=0}^\infty\exp\bigg(-\frac{1}{2}\nu_1(\beta(t))\bigg)\sum_{j=0}^n{m \choose j}(e^{-\tilde{\mu_1}t})^j(1-e^{-\tilde{\mu_1}t})^{m-j}\frac{(\frac{1}{2}\nu_1\beta(t))^{n-j}}{(n-j)!}\frac{(\frac{1}{2}\omega_k)^m e^{-\frac{1}{2}\nu_1\beta(t))^{n-j}\omega_k}}{m!}$$ 

$$=e^{\frac{-1}{2}(\nu_1\beta(t)+\omega_k)}\sum_{m=0}^\infty\sum_{j=0}^m{m \choose j}(e^{-\tilde{\mu_1}t})^j(1-e^{-\tilde{\mu_1}t})^{m-j}\frac{1}{m!(n-j)!}(\frac{1}{2}\omega_k)^m(\frac{1}{2}\nu_1\beta(t))^{n-j}$$

$$=\frac{1}{n!}e^{1/2(\beta(t)+\omega_k)} \sum_{j=0}{n\choose j}(\frac{1}{2}\omega_k e^{-\tilde{\mu}_1})^j(\frac{1}{2}\beta(t))^{n-j}\sum_{v=0}^\infty\frac{[\frac{1}{2}\omega_k e^{-\tilde{\mu}_1t}]^v}{v!}  $$
$$\frac{1}{n!}[\frac{1}{2}(\beta(t)+\omega_k e^{-\tilde{\mu}_1t})]^n exp(-(\beta(t)+\omega_k e^{-\tilde{\mu}_1}t) ) $$
Por convolución $$P(W_k(t)=n)=P(M_k(t)+F_k(t))$$
Esto quiere decir que $$W_k(t)=n\sim Poisson(\beta+\omega e^{\tilde{\mu}_1 t})$$
Ahora se estudiará las siguientes distribuciones de probabilidad.
Se define: $$\hat{M}_k(t)=M_k(t)-\gamma_k(t)$$
$$\hat{F}_k(t)=F_k(t)-\gamma_k(t)$$
$M_k(t)$, $F_k(t)$ pueden ser interpretados como el número de parásitos machos y hembras, respectivamente, sin pareja en el huésped k.
Además $$\gamma_k(t)=\frac{1}{2}(W_k(t)-\hat{M}_k(t)-\hat{F}_k(t) )$$
\begin{Lem}
    Dado $n\in\N,$
    $$\{\hat{M}_k(t)\}=\bigcup_{j=0}^\infty \{F_k(t)=j\}\cap\{M_k(t)=j+n\}$$
    \begin{proof}
        Dado $\omega\in\bigcup_{j=0}^\infty \{F_k(t)=j\}\cap\{M_k(t)=j+n\}$, $\exists j\in\N$ tal que $$\omega\in\{F_k(t)=j\}\quad y \quad\omega\in\{M_k(t)=j+n\}$$
        Entonces $$\gamma_k(t)(\omega)=F_k(t)(\omega)=j<M_k(t)(\omega)=j+n$$
        $$M_k(t)(\omega)=\gamma_k(t)(\omega)+n$$
        $$\hat{M}_k(t)(\omega)=n$$
        Esto es $\omega\in\{\hat{M}_k(t)=n\}$\\
        Dado $\omega\in\{\hat{M}_k(t)=n\}=\{M_k(t)-\gamma_k(t)=n\}$\\$$M_k(t)(\omega)-\gamma_k(t)(\omega)=n$$
        como $n\geq 1$, $\gamma_k(t)$ entonces $$\hat{F}_k(t)=0\quad y \quad \gamma_k(t)=F_k(t)$$
        i $F_k(t)(\omega)=j\in\N$
        $$M_k(t)(\omega)=\hat{M}_k(t)(\omega)+\gamma_k(t)(\omega)=n+j$$
        Entonces $\omega\in\{F_k(t)=j\}\cap\{M_k(t)=n+j\}$
    \end{proof}
\end{Lem}
\label{LEMAIMPORTANTE}
\begin{Lem}
    Dado $n\in\N,$ entonces para cada k, $1\leq k\leq N_1$
    \begin{eqnarray}
        P(\hat{M}_k(t)=n)=e^{-\beta(t)}\sum_{\iota=0}^\infty\sum_{m=0}^\infty p_{\iota}^{(k)} q_m^{(k)}\sum_{i=0}^\iota\sum_{j=0}^m{\iota\choose i}{m\choose j} (e^{-\tilde{\mu}_1t})^{i+j}(1-e^{-\tilde{\mu}_1t})^{\iota+m-i-j}I_{n+i-j}(\beta(t))
        \label{etiqueta1}
    \end{eqnarray}
    \begin{eqnarray}
        P(\hat{F}_k(t)=n)=e^{-\beta(t)}\sum_{\iota=0}^\infty\sum_{m=0}^\infty p_{\iota}^{(k)} q_m^{(k)}\sum_{i=0}^\iota\sum_{j=0}^m{\iota\choose i}{m\choose j} (e^{-\tilde{\mu}_1t})^{i+j}(1-e^{-\tilde{\mu}_1t})^{\iota+m-i-j}I_{n+j-i}(\beta(t))
        \label{etiqueta2}
    \end{eqnarray}
    Donde $I_n$ denota la función de Bessel modificada de primera especie de orden n.
    \begin{proof}
        Usando el Lema \ref{LEMAIMPORTANTE} y la independencia de $F_k(t)$ y $M_k(t)$ se sigue que:
        $$P(\hat{M}_k(t)=n)=\sum_{j=0}^\infty P(F_k(t)=j)P(M_k(t)=j+n)$$
        $$=\sum_{j=0}^\infty \big(\sum_{\iota=0}^\infty p_{\iota}^{(k)} P_{\iota ,j}(0,t) \big)\big(\sum_{m=0}^\infty q_m^{(k)}P_{m,j+n}(0,t)\big)$$
        Donde:
        $$P_{\iota ,j}(0,t)=\exp\big(-\frac{1}{2}\nu_1(\beta(t)\big)\sum_{i=0}^n{\iota \choose i}(e^{-\tilde{\mu_1}t})^i(1-e^{-\tilde{\mu_1}t})^{\iota-i}\frac{\big(\frac{1}{2}\nu_1(\beta(t)\big)^{\iota-i}}{(\iota-i)!}$$
        $$P_{m ,j+n}(0,t)=\exp\big(-\frac{1}{2}\nu_1(\beta(t)\big)\sum_{i=0}^n{m \choose i}(e^{-\tilde{\mu_1}t})^i(1-e^{-\tilde{\mu_1}t})^{m-i}\frac{\big(\frac{1}{2}\nu_1(\beta(t)\big)^{j+n-i}}{(j+n-i)!}$$
        La formula del lema sigue reemplazando estos valores en \ref{etiqueta1} y si se intercambian los role de $p_j^{(k)}$ y $q_j^{(k)}$ .Se obtiene \ref{etiqueta2} intercambiando los roles de $p_j^{k}$ y $q_m^{k}$
    \end{proof}
\end{Lem}