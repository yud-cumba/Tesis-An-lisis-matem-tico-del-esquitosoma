\section{Ecuaciones de Kolgomorov}
\begin{comment}
    Usando la ecuación de Chapman-Kolgomorov se obtiene que
    $$P_{m,n}(s,t)=\sum_{k=0}^\infty P_{m,k}(s,u)P_{k,n}(u,t),$$
    $$\Pi_{i,j}(s,t)=\sum_{k=0}^{N_2} \Pi_{i,k}(s,u)\Pi_{k,j}(u,t),$$
\end{comment}

%%%%%%%%%%%%%%%%%%%
En esta sección se investigará algunas de las consecuencias de asumir que el problema inicial de Kolgomorov tiene solución. Los resultados obtenidos tienen un valor especial en un punto de vista epidemiológico. Se muestra un método para encontrar la solución de este problema resolviendo un problema con valores iniciales de una ecuación diferencial de primer orden, de esta forma podremos investigar alguna consecuencia de asumir que nuestro problema de valor inicial de Kolmogorov tiene solución.

Dado $m\in\N$, $s\geq 0$, consideramos para cada variable aleatoria $M_k(t)$, $k=1,2,\ldots,N_1$ $t\geq 0$ una distribución de probabilidad $\{p_n(t)\}_{n\in\N}$, tal que $p_n(t)=P_{m,n}(s,t)$. La función generadora de probabilidad ($f.g.p.$) $G$ asociada a la distribución $\{p_k(t)\}$) es
\begin{equation}
    G(t,z)=\sum_{n=0}^\infty z^n p_n=\sum_{n=0}^\infty z^n P_{m,n}(s,t)\label{tesis-funcGeneradoraDeM}
\end{equation}
Entonces,
\begin{eqnarray}
    \frac{\partial G(t,z)}{\partial z}=\sum_{n=1}^\infty n z^{n-1}p_n(t)=\sum_{n=0}^\infty (n+1)z^{n}p_n(t)
    \frac{\partial G(t,z)}{\partial t} = \sum_z^n \frac{\partial p_n(t)}{\partial t}
\end{eqnarray}
\begin{eqnarray}
    \frac{\partial G(t,z)}{\partial t}=\sum_{n=0}^\infty z^n p_n'(t)=p'_0(t)+\sum_{n=1}^\infty z^n p_n'(t),
\end{eqnarray}
Además, (\ref{tesis-edo-p_n}) nos muestra que $$p'_n(t)=-\big(\frac{1}{2}\nu_1E[S(t)]+\tilde{\mu}_1\big)p_n(t)+\frac{1}{2}\nu_1E[S(t)]p_{n-1}(t)+\mu_1 (n+1)p_n+1(t) \quad n\in\N$$
$$p'_0(t)=-\frac{1}{2}\nu_1E[S(t)] p_0(t)+\mu_1 p_1(t)$$
Cuyo valor inicial vendría a ser $$p_n(s)=\delta_{m,n}$$\\
Si denotamos $Y(t)=E[S(t)]$ y sustituimos los valores de $p'_n(t)$ obtenemos que
\begin{eqnarray*}
    \frac{\partial G(t,z)}{\partial t} =-\frac{1}{2}\nu_1Y(t)p_0(t)+\mu_1 p_1(t)+\sum_{n=1}^\infty z^n\big[-\big(\frac{1}{2}\nu_1Y(t)+\mu_1)p_n(t)+\frac{1}{2}\nu_1Y(t)p_{n-1}(t)+\mu_1(n+1)p_{n+1}(t)\big]
\end{eqnarray*}
\begin{eqnarray*}
    \quad\quad\quad=-\frac{1}{2}\nu_1Y(t)\sum_{n=1}^\infty p_n(t)z^n+\mu_1\sum_{n=0}^\infty (n+1)p_n(t)z^n-\mu_1\sum_{n=1}^\infty p_n(t)z^n+\frac{1}{2}\nu_1Y(t)\sum_{n=1}^\infty p_n(t)z^n
\end{eqnarray*}
\begin{eqnarray*}
   \quad\quad\quad=-\frac{1}{2}\nu_1Y(t)G+\mu_1\Gz-\mu_1z\Gz+\frac{1}{2}\nu_1Y(t)z G=\frac{1}{2}\nu_1Y(t)(z-1)G-\mu(z-1)\Gz
\end{eqnarray*}
Finalmente, obtenemos la siguiente ecuación diferencial parcial
$$ \Gt+\mu(z-1)\Gz=\frac{1}{2}\nu_1Y(t)(z-1)G$$
con condición inicial
$$G(s,z)=\sum_{n=0}^\infty z^n p_n(s)=z^m$$