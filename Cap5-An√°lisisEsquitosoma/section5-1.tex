\section{Sistema de Ecuación diferencial ordinaria de Kolgomorov}
Por los postulados expuestos en la sección (\ref{datos-generales}), para $t\geq 0$  $M_k(t), F_k(t)$ y $S(t)$ son procesos de nacimiento y muerte, de los cuales consecuentemente, los sistemas de ecuaciones diferenciales asociados a su respectiva distribución $\{P_{m,n}(s,t)\}_{n=1}^\infty$
% y $\{\Pi_{i,j}(s,t)\}_{j=1}^\infty$ 
de acuerdo a la demostración expuesta en (\ref{procNacimiento-edo-n}) referido al proceso de nacimiento y muerte son
\begin{eqnarray}
    \begin{array}{cr}
        \frac{P_{m,n}(s,t)}{\partial t} = & -\bigg[\frac{1}{2}\nu_1E[S(t)]+\mu_1\bigg]P_{m,n}(s,t)+\frac{1}{2}\nu_1E[S(t)]P_{m,n-1}(s,t) \\
        & +\mu_1 (n+1)P_{m,n+1}(s,t)
    \end{array}
    \label{tesis-edo-p_n}
\end{eqnarray}
% \begin{eqnarray}
%     \begin{array}{cr}
%      \frac{\partial\Pi_{i,j}(s,t)}{\partial t}= & -\bigg[\nu_2(N_2-j)E\big[\sum_{k=1}^{N_1}\gamma(t)\big]+\mu_2\bigg]\Pi_{i,j}(t)
%      +\tilde{\mu}_1 (j+1)\Pi_{i,j+1}(t)
%      \\&+\nu_2(N_2-j)E\big[\sum_{k=1}^{N_1}\gamma(t)\big]\Pi_{i,j-1}(t)
%     \end{array}
% \end{eqnarray}
para $m,i\in\N$, $s\geq 0$.\\
Por lo tanto si $n\not=0$, $$P\big(M_k(s)=n |\thinspace M_k(s)=m\big)=0\quad $$
% \quad \Pi\big(M_k(s)=n |\thinspace M_k(s)=m\big)=0,$$
mientras que si $n=m$, $$P\big(M_k(s)=m |\thinspace M_k(s)=m\big)=1\quad $$
% y \quad \Pi\big(M_k(s)=m |\thinspace M_k(s)=m\big)=0.$$\\
Por ello es conveniente extender la definición de $P_{m,n}$ y
% $\Pi_{i,j}$ 
estableciendo $$P_{m,n}(s,s)=\delta_{m,n}\quad m,n=0,1\cdots, s\geq 1$$
% $$\Pi_{i,j}(s,s)=\delta_{i,j}\quad i,j=0,1\cdots,N_2, s\geq 0$$
donde 
$$\delta_{h,k}=
    \begin{cases}
    1, & \mbox{ $h=k$ } \\
    0, & \mbox{ $h\not=k$},
    \end{cases}$$
las cuales llegarían a ser las condiciones iniciales de nuestros sistemas de ecuaciones diferenciales.
Esto quiere decir que si el sistema $M_k$ o $F_k$ se encuentra en el estado $m$ en la época $s$ y no transcurre ningún salto de tiempo (es decir permanece en el estado $s$) el sistema seguirá permaneciendo en el estado $m$, ya que si no transcurre tiempo, tampoco ocurrirá cambio alguno.\\
Por la independencia de las distribuciones de transición, a esta sistema de ecuaciones se le conoce como problema de valor inicial de Kolmogorov.\\ El teorema de Daniel-Kolgomorov expuesta en la sección $(2.5)$ nos muestra que efectivamente existen cadenas de Markov $M_k$, $F_k$, y $S$ que cumplen las propiedades de los postulados si y solo existe solución a las ecuaciones de Chapman-Kolmogorov.\\
\begin{Obs}
    Claramente, las suposiciones realizadas solo representan una aproximación de lo que representan una infección real. Se ignora la infección específica por edad o sexo en la población humana.Las tasas de mortalidad son independientes de la edad y de la densidad de población. Hubiera sido más natural considerar a $P_{m,m+1}(t,t+h)$ 
    % y $\Pi_{i,i+1}(t,t+h)$
    proporcional a $S(t)$ y a $\sum_{k=1}^{N_1}\gamma_k(t)$ respectivamente, pero resulta más conveniente reemplazar estas cantidades por sus esperanzas.\\
    Además, estos postulados no tienen en cuenta los efectos de edad en la tasa de oviposición de un esquistosoma femenino emparejado.Tampoco hemos considerado el posible desarrollo de resistencia a la infección en los huéspedes o los períodos latentes durante los cuales las cercarias y las esquistosomas se desarrollan hasta sus formas maduras.\\A pesar de estas simplificaciones y omisiones, confiamos en que nuestras hipótesis retratan principales
    características de las relaciones huésped-parásito con suficiente similitud para producir útiles conclusiones cualitativamente confiables.
\end{Obs}