\section{Curvas integrales de campos vectoriales}
\begin{Def}
    Se denomina dominio a un subconjunto $\Omega$ de $\R^n$ abierto y conexo. Denotaremos por $\partial\Omega$ a la frontera de $\Omega$.
\end{Def}
Una relación es llamada ecuación diferencial cuasilineal si tiene la forma
\begin{eqnarray}
    \label{ecuacuasi}
    a(x,y,u)\ux+b(x,y,u)\uy=c(x,y,u)
\end{eqnarray}
donde se asume que a,b,c son funciones reales de clase $C^1(\Omega)$
Una ecuación semilineal será un caso particular de \ref{ecuacuasi} si tomamos $a(x,y,u)=a(x,y)$ y $b(x,y,u)=b(x,y)$
\begin{eqnarray}
\label{ecuasemi} 
a(x,y)\ux+b(x,y)\uy=c(x,y,u) 
\end{eqnarray}
Sea $V:\Omega\subset\R^3\rightarrow\R^3$ $V(x,y,z)=(a(x,y),b(x,y),c(x,y))$
Se asumirá que 
\begin{enumerate}
    \item  $a,b,c$ no se anulan simultáneamente en algún punto de $\Omega$
    \item $a,b,c\in C^1(\Omega)$
\end{enumerate}
\begin{Def}
    Una a curva $\gamma\subset\Omega$ es una curva integral del campo V si tiene como vector tangente a $V=V(x,y,z)$ en cada uno de sus puntos. Si $\gamma$ es parametrizado por algún $s\in I$
    $\gamma(s))=(x((s),y(s),z(s))$ que tiene como vector tangente $V=V(x(s),y(s),z(s))$ para cada $s\in I$.
    Explícitamente se cumple la relación 
    \begin{eqnarray}
        x'(s)=a(x(s),y(s),z(s)) , y'(s)=b(x(s),y(s),z(s)) , z'(s)=b(x(s),y(s),z(s))\label{derivadacurva}
    \end{eqnarray}
    Una solución $(x (t), y (t), z (t))$ del sistema anterior, definida para s en algún intervalo I, puede ser considerado como una curva en $\Omega$. Llamaremos a esta curva una curva solución de la
    sistema \ref{derivadacurva}. Cada curva de solución del sistema es un
    curva integral del campo vectorial V.
\end{Def}