\section{Soluciones generales y condiciones auxiliares}
\begin{Def}
    Una ecuación diferencial en derivadas parciales, puede describirse como una relación donde aparece una función incógnita $\mu$ junto con al menos una derivada parcial. Dado que en una ecuación diferencial parcial deben aparecer derivadas parciales, se sobreentiende que $\mu$ depende de al menos dos variables independientes. En general, es una relación de la forma :
    $$F(x_1,\ldots,x_n, u, u_{x_1},\ldots u_{x_n},\ldots,u_{x^{m_1}_1},\ldots u_{x^{m_k}_n} )=0$$
    Donde $n,m,k \in\N$ y $m_1+\ldots+m_k<+\infty$ y $u_{x^{m}_i}=\frac{\partial^{m} u}{\partial x_i^m}$ la derivada parcial de orden $m$ de $u$ respecto a $x_i$.
\end{Def}
\begin{Def}
    Dada una ecuación diferencial parcial, se denomina solución clásica de una ecuación diferencial parcial a una función que satisface la ecuación y que posee todas las derivadas parciales (involucradas en la ecuación) continuas.
\end{Def}
\begin{Obs}
    Se suele denotar $u_x$ en lugar de $\ux(x,y)$ para simplificar la notación.
\end{Obs}
Dada una ecuación diferencial parcial $F(x,u,D^{\alpha}_u)=0$ diremos que u es solución general de la ecuación diferencial parcial si contiene como casos particulares a cualquier otra solución de la ecuación diferencial parcial.
\begin{Ejm}
    Consideremos la ecuación diferencial parcial en dos variables.
    $$\ux(x,y)=0$$
    Entonces la solución general es $u(x,y)=f(y)$ para alguna función $f\in C^1(\R)$\\
    Si adicionamos condiciones adicionales a la ecuación tendremos una solución más precisa. Estas condiciones adicionales pueden provenir de las propiedades físicas, químicas, etc. del modelo que da origen la ecuación o a la naturaleza del dominio sobre el que queremos estudiar el problema.
\end{Ejm}
Las ecuaciones diferenciales parciales que dependen de una variable temporal son llamadas ECUACIONES DE EVOLUCIÓN. Por ejemplo, la ecuación del calor  $$\ut-\uxx=f$$ y las ondas $$\utt-\uxx=f$$ son ejemplos de ecuaciones de evolución.\\Cuando imponemos una condición sobre la solución de una ecuación diferencial parcial para un valor dela variable temporal, esta condición se denomina CONDICIÓN INICIAL de la ecuación diferencial parcial.