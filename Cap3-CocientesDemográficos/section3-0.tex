\section{Flujo y stock}
\begin{Def}
    Una población es un conjunto renovable de individuos que cumplen determinada condición, sean humanos, animales o cosas y no son un conjunto estático, sino que están sometidas a un proceso continuo de cambio, por salidas y entradas de individuos en dicha población.
\end{Def}
"Los indicadores estadísticos utilizados en la demografía rara vez son demasiado complejos. De hecho, en la mayoría de los casos solo puede haber una relación entre las dos cantidades. Este caso particular de un indicador en el que se vinculan dos cantidades se denomina cociente y existen diferentes tipos según la naturaleza de los datos que se utilicen. Para entender lo siguiente, es importante entender primero qué significa la diferencia entre flujo y stock".
\begin{Def}
    "El stock es el número de artículos en un momento determinado. \\
    Tienen la dimensión del tiempo instantáneo. Se refieren a las existencias en un momento determinado". \\
    Los datos de stocks poblacionales pueden tener como fuente los censos, padrones, estimaciones de población o encuestas.\\
    Un ejemplo de stock sería el número de personas vivas el 25 de Diciembre del 2019.
\end{Def}
\begin{Def}
    Los flujos son eventos o fenómenos (como el nacimiento o la muerte) en un espacio-tiempo continuo. \\ Los datos de transmisión se toman del "registro" de eventos como las estadísticas vitales de nacimiento, muerte y matrimonio. \\ \
    Un ejemplo de flujo sería el número de nacidos vivos durante el año 2019.
\end{Def}
