\section{Algunas propiedades elementales}

La importancia esencial de la aplicación de los métodos de cálculo de la probabilidad reside en su capacidad para estimar o predecir eventos. Cuanto mayor sea la cantidad de datos disponibles para calcular la probabilidad de un acontecimiento, más preciso será el resultado calculado. Dada la complejidad de los sistemas en los que suele aplicarse la teoría de la probabilidad, se requiere de ciertas propiedades que faciliten su cálculo.
Las demostraciones de los resultados expuestos en este capítulo se pueden encontrar en \cite{intro-probabilidad}, \cite{Feller},\cite{Rincon1}, \cite{Rincon2}.
\begin{Prop}
    Sea $(\Omega,\thinspace\mathscr{F},P)$ un espacio de probabilidad, $A,B$ conjuntos tal que, $A,B\subset\Omega$, se cumple que:
    \begin{enumerate}
        \item $P(A^c)=1-P(A)$.
        \item $P(\emptyset)=0$.
        \item Si $A\subseteq B$ entonces $P(A)\leq P(B)$.
        \item Si $A\subseteq B$ entonces $P(B-A)=P(B)-P(A)$.
    \end{enumerate}
\end{Prop}
\begin{Def}
    Sea $\Omega$ el espacio muestral de un experimento aleatorio. Decimos que la colección de eventos $\{B_1,B_2,\ldots,B_n\}$ es una partición finita de $\Omega$ si se cumplen las siguientes condiciones:
    \begin{enumerate}[a)]
        \item $B_i\subset\Omega$ $i=1,2,\ldots,n$.
        \item $B_i\not=\emptyset$, $i=1,2,\ldots,n$.
        \item $B_i\cap B_j=\emptyset$ , para $i\not= j$.
        \item $\bigcup_{i=1}^n B_i=\Omega$.
    \end{enumerate}
\end{Def}
El siguiente resultado es bastante útil, pues establece un método para calcular las probabilidades. Mediante este método, dado un evento $A$, cuya probabilidad se busca, se trata de encontrar una partición de $A$ donde cada evento de esta partición será más simple de calcular.
\begin{Prop}
    Sea $(\Omega,\mathscr{F},P)$ un espacio de probabilidad, $\{A_k\}_{k=1}^n$ una partición finita de $\Omega$, se tiene que
    $$P(\bigcup_{k=1}^n A_k)=\sum_{k=1}^n P(A_k)$$ 
\end{Prop}
\begin{Cor}
    Si $(\Omega,\mathscr{F},P)$ es un espacio de probabilidad, $A$ y $B$ eventos, entonces
    $$P(A\cup B)=P(A)+P(B)-P(A\cap B)$$
\end{Cor}
En algunos casos, nos topamos con una colección de eventos que tienen cada uno la misma probabilidad de que sucedan. A estos eventos se le conocen como equiprobables.
Dado $n,N\in\N$, tal que $n<N$, $\Omega=\{\omega_1,\ldots,\omega_N\}$ un espacio muestral y $A=\{\omega_1,\ldots,\omega_n\}$ un evento del espacio muestral $\Omega$.
Si suponemos que los conjuntos  $\{\omega_k\}_{k=1}^N$, son equiprobables, entonces $P(\{\omega_i\})=P(\{\omega_j\})$ para $i,j=1,2,\ldots,N$, y como consecuencia $$1=P(\bigcup_{k=1}^N \{\omega_k\})=\sum_{k=1}^N P(\{\omega_k\})=N P(\omega_k).$$
\begin{eqnarray}
    \label{probabilidadn}
    P(\omega_k)=\frac{1}{N}\quad k=1,2,\ldots ,N.
\end{eqnarray}
Además
$$P(A)=P(\bigcup_{k=1}^n \{\omega_k\})=\sum_{k=1}^n P(\{\omega_k\})=n P(\{\omega_k\})$$
Usando la expresión (\ref{probabilidadn})
$$P(A)=\frac{n}{N}.$$
Este método de encontrar probabilidades, basándose en la equiprobabilidad de los posibles resultados, es conocido como la definición clásica de probabilidad.\\
Históricamente, esta forma de calcular probabilidades es una de las primeras en utilizarse; se aplicó con bastante éxito en problemas de juegos de azar y ayudó a sentar las bases para construir la teoría matemática.
\begin{Def}
    Sea $A$ un subconjunto de un espacio muestral $\Omega$ de cardinalidad finita. Si $\#A$ denota la cardinalidad del conjunto $A$. Se define la probabilidad clásica del evento $A$ como el cociente $$P(A)=\frac{\#A}{\#\Omega}$$
    \label{defprobClásica}
\end{Def}
\begin{Ejm}
    El ejemplo típico de equiprobabilidad se presenta al considerar el experimento  en lanzar un dado no trucado sobre una mesa y observar el número que muestra finalmente su cara superior. Considere el experimento aleatorio de lanzar un dado equilibrado.\\ El espacio muestral es el conjunto  $\Omega=\{1,\thinspace 2\thinspace 3,\thinspace 4,\thinspace 5,\thinspace 6\}$. Si deseamos calcular la probabilidad (clásica) del evento A, correspondiente a obtener un número par, es decir, la probabilidad de $A=\{2,\thinspace 4,\thinspace 6\}$, entonces $$P(A)=\frac{\#\{2,\thinspace 4,\thinspace 6\}}{\#\{1,\thinspace 2\thinspace 3,\thinspace 4,\thinspace 5,\thinspace 6\}}=\frac{3}{6}=0.5$$
\end{Ejm}
En ocasiones un experimento aleatorio depende de otros experimentos, también aleatorios, los cuales se realizan uno después del otro.\\
Como la probabilidad está ligada a nuestra ignorancia sobre los resultados de la experiencia, el hecho de que ocurra un suceso, puede cambiar la probabilidad de los demás.
\begin{Def}
    Sea $(\Omega,\mathscr{F},P)$ un espacio de probabilidad, $A,B\subset\Omega$ tal que $P(B)>0$. La probabilidad condicional de algún evento $A$, dado que el evento $B$ suceda previamente, es una función que se denota $P(\cdot\thinspace|\thinspace B):\Omega\rightarrow [0,1]$ y se define como $$P(A\thinspace|\thinspace B)=\frac{P(A\cap B)}{P(B)}$$
\end{Def}
Cuando ocurre un suceso antes de realizar otro experimento, se reduce el espacio muestral y es por eso que cambia la probabilidad. A veces es más fácil calcular la probabilidad condicionada teniendo en cuenta este cambio de espacio muestral.
No tiene por qué haber una relación causal o temporal entre $A$ y $B$. A puede preceder en el tiempo a $B$, sucederlo o pueden ocurrir simultáneamente. $A$ puede causar $B$, viceversa o pueden no tener relación causal. Las relaciones causales o temporales son nociones que no pertenecen al ámbito de la probabilidad. Pueden desempeñar un papel o no, dependiendo de la interpretación que se le dé a los eventos.
Siendo la probabilidad condicional una probabilidad calculada en un espacio muestral reducido (el cual sería $B$), es de esperarse que ésta tenga las mismas propiedades que cualquier medida de probabilidad.
\begin{Prop}
    $P(\cdot\thinspace|\thinspace B)$ es una medida de probabilidad en el espacio de medida $(\Omega,\mathscr{F})$
\end{Prop}
Gracias a esta propiedad tenemos una útil herramienta para calcular probabilidades conocida como la regla del producto, la cual se puede utilizar para determinar la probabilidad de la intersección de dos eventos. Esta ley se deriva de la definición de probabilidad condicional.
$$P(A\cap B)=P(A\thinspace|\thinspace B)P(B)$$
De la misma manera, en los casos en que la ocurrencia de un evento $A$ no altere la probabilidad de otro evento $B$, se puede hablar de independencia de probabilidad.
\begin{Def}
    Se dice que los eventos $A, B$ son independientes si se cumple la igualdad $P(A\cap B)=P(A)P(B)$. Bajo la hipótesis adicional de $P(B)>0$. La condición de independencia puede escribirse como $P(A\thinspace|\thinspace B)=P(A)$.
\end{Def}
 Esto intuitivamente significa que la ocurrencia del evento $B$ no afecta al evento $A$.\\
El siguiente resultado es bastante útil cuando se quiere determinar la probabilidad de algún conjunto grande, bastará con hallar las probabilidades de conjuntos más pequeños y probabilidades condicionales con respecto a esos conjuntos
\begin{Teo}
    Sea $\{B_1,B_2,\ldots,B_n\}$ una partición de $\Omega$ tal que $P(B_i)\not=0$, para  $i=1,\ldots,n$, para cualquier evento $A$, $$P(A)=\sum_{i=1}^n P(A\thinspace|\thinspace B_i)P(B_i)$$
\end{Teo}

