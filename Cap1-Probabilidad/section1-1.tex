\section{Conceptos generales}
    En la naturaleza se pueden encuentran dos tipos de fenómenos o experimentos: deterministas y aleatorios. Un experimento determinista es aquel cuyo resultado es el mismo si se la repite bajo las mismas condiciones, como por ejemplo, el medir el volumen de un gas en teoría produce siempre el mismo resultado, cuando la presión y la temperatura son constantes.\\
    En contraste a esto, un experimento aleatorio es aquel que aunque se repitan bajo las mismas condiciones, el resultado no será siempre el mismo.\\
    Los conceptos y teoremas expuestos en esta sección fueron tomados de \cite{Rincon1} \cite{Rincon2}.
\begin{Ejm}
    Algunos ejemplos de experimentos deterministas:
    \begin{enumerate}[a)]
        \item Medir el volumen de un gas cuando la presión y la temperatura son constantes.
        \item Calcular la distancia que recorre un carro en $1$ hora a una velocidad constante.
    \end{enumerate}
\end{Ejm}
\begin{Ejm}
\label{Prob-Ejm-Aleatorio}
Algunos ejemplos de experimentos aleatorios:
    \begin{enumerate}[a)]
        \item Registrar el número de accidentes que ocurren en una determinada calle de una ciudad.
        \item Registrar si una mujer con  determinada enfermedad la trasmite a alguno de sus 3 hijos.
    \end{enumerate}
\end{Ejm}
Cuando se tratan de experimentos aleatorios, en principio no sabremos cuál será el resultado obtenido. El experimento asociado a nuestro fenómeno, da lugar a un solo resultado (que denotaremos $\omega$) de entre un conjunto de posibles resultados.
\begin{Def}
Un espacio muestral es un conjunto arbitrario $\Omega$ que tiene como objetivo agrupar todos los posibles resultados del experimento aleatorio en cuestión.
\end{Def}
\begin{Ejm}
Los espacios muestrales respecto a los experimentos aleatorios descritos en el ejemplo (\ref{Prob-Ejm-Aleatorio}) serían:
\begin{enumerate}[a)]
    \item $\Omega=\{0,1,2,3,\ldots\}$
    \item $\Omega=\{\textit{true, false}\}$
\end{enumerate}
\end{Ejm}
De manera intuitiva, los subconjuntos de resultados (es decir del conjunto $\Omega$) que comparten alguna característica en común reciben el nombre de sucesos aleatorios.
Estos subconjuntos recibirán los nombres de sucesos si cumplen determinadas condiciones. Cuando el resultado de un experimento realizado pertenece a un suceso específico $A$, decimos que "$A$ ha ocurrido o se ha realizado".
\begin{Def}
    Una clase o colección $\mathscr{F}$ de subconjuntos de $\Omega$ es un $\sigma- \textit{álgebra}$ si cumple que:
    \begin{enumerate}
        \item $\Omega\in\mathscr{F}$
        \item Si $A\in\mathscr{F}$ entonces $A^c\in\mathscr{F}$
        \item Si $\{A_n\}_{n\in\N}\in\mathscr{F}$, entonces  $\bigcup_{k=1}^\infty A_k\in\mathscr{F}$
    \end{enumerate}
\end{Def}
 Formalmente, a los elementos de un $\sigma-$álgebra $\mathscr{F}$ se los llama eventos, sucesos aleatorios o conjuntos medibles.
 Veamos un ejemplo que nos ayuda a ver intuitivamente el concepto de $\sigma-$álgebra.
\begin{Ejm}
    Supongamos que queremos registrar el número final de las placas de los carros que circulan en una determinada calle de una ciudad, para poder aplicar la regla de pico y placa. Para este caso, el espacio muestral de todos los posibles números finales de una placa arbitraria serían 
    $$\Omega=\{0,\thinspace 1,\thinspace 2,\ldots 9\}.$$
    Sea $\omega$ es resultado del experimento, sea cual su valor, $\omega \in\Omega$, por tanto, $\Omega$ debe ser un evento; al cual llamaremos evento seguro. Esto es $\Omega\in\mathscr{F}.$\\ Definamos el evento $A=\{0,\thinspace 2,\ldots ,8\}.$ Esto se interpreta al suceso de que la placa del auto arbitrario termine en número par. Si tiene perfecto sentido pensar en que el resultado $\omega$ pueda ser un número par (es decir $\omega\in A$), también tiene perfecto sentido pensar que este podría ser un número impar (es decir que $\omega \in A^c$) es, por tanto, razonable exigir que $A^c$ sea un evento si $A$ lo es. Esto es $A^c\in\mathscr{F}$ si $A\in\mathscr{F}.$\\Si definimos el evento $B=\{1\}$ que ocurre cuando la placa acabe en el número $1$, tiene sentido pensar que escogiendo un auto arbitrario, el resultado $\omega$ podría ser un par ($\omega\in A$) o quizá el resultado podría ser $1$ ($\omega\in B$). Por ser ambos eventos, puede ocurrir que $\omega\in A\cup B$, por lo tanto exigiremos que la unión finita de sucesos sea un suceso. Esto es si $A,\thinspace B\in\mathscr{F}$ entonces $A\cup B\in \mathscr{F}.$
\end{Ejm}
Menos intuitiva es la exigencia de estabilidad frente a uniones numerables, la respuesta más
convincente es que, de ese modo, se obtiene una teoría matemática más rica.
\begin{Def}
    Si $\mathscr{C}$ es una colección no vacía de subconjuntos de $\Omega$.\\La $\sigma-\textit{álgebra}$ generada por $\mathscr{C}$ denotada por $\sigma(\mathscr{C})$ es $$\sigma(\mathscr{C})=\bigcap\{\mathscr{F}|\thinspace\mathscr{F} \textit{ es un }\sigma-\textit{álgebra y }  \mathscr{C}\subset\mathscr{F}\}$$
    Este conjunto es la menor $\sigma-$álgebra que contiene a $\mathscr{C}$.
\end{Def}
\begin{Ejm}
    El conjunto $\mathscr{P}(\Omega)$ de todos los subconjuntos de $\Omega$ es trivialmente un $\sigma$-álgebra en $\Omega$ a la cual llamaremos discreta. \\ Llamaremos espacio medible discreto a un conjunto numerable provisto de esta $\sigma-$álgebra. En lo que sigue, cualquier conjunto numerable se supondrá provisto de la $\sigma-$álgebra discreta, a menos que se indique lo contrario.
\end{Ejm}
\begin{Ejm}
    El conjunto $\{\Omega, \emptyset\}$ es la más pequeña $\sigma-$álgebra que podemos considerar de $\Omega$. Se llamará $\sigma-$álgebra trivial.
\end{Ejm}
\begin{Def}
    \label{def-medidaProbabilidad}
    Sea $\mathscr{F}$ una $\sigma-$álgebra del espacio muestral $\Omega$, $k\in\N$ $A_k$, sucesos asociados a $\mathscr{F}$.
    Una función $P:\mathscr{F}\rightarrow [0,1]$ es una medida de probabilidad si cumple que $P(\Omega)=1$ y P es $\sigma-aditiva$, es decir que se cumple que $$P\big(\bigcup_{k=1}^\infty A_k\big)=\sum_{k=1}^\infty P(A_k)$$
    donde $A_i\cap A_j=\emptyset$, para $i\not= j$
\end{Def}
El número $P(A)$ es una forma de medir la posibilidad que suceda el evento $A$ al efectuar una vez el experimento aleatorio.\\
Una probabilidad cercana a $0$ indica que es poco probable que ocurra un evento, mientras que una probabilidad cercana a $1$ indica que es casi seguro que éste ocurra.\\
La probabilidad es importante en la toma de decisiones debido a que proporciona una manera de medir, expresar y analizar las incertidumbres asociadas con eventos futuros.
\begin{Def}
    Sea $\Omega$ un espacio muestral, $\mathscr{F}$ un $\sigma-$álgebra de $\Omega$, $P$ una medida de probabilidad.\\
    A la terna $(\Omega,\thinspace\mathscr{F})$ se le llama espacio de medida o espacio medible y a la terna $(\Omega,\mathscr{F},P)$ se le conoce como espacio de probabilidad.
\end{Def}