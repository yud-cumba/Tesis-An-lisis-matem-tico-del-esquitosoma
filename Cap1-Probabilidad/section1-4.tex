\section{Algunas distribuciones de una variable aleatoria}
De forma intuitiva, una variable aleatoria puede concebirse como un valor numérico que está afectado por el azar. Por ejemplo, en una epidemia de alguna enfermedad, se sabe que una persona cualquiera puede enfermar o no (estos serían los sucesos), pero no se sabe cuál va a ocurrir. Solamente se puede decir que existe una probabilidad de que la persona enferme y otra, de que no enferme. Dependiendo de las circunstancias, estos sucesos no siempre serán equiprobables.\\
Para trabajar de manera sólida con variables aleatorias en general es necesario considerar un gran número de experimentos, para su tratamiento estadístico, cuantificar los resultados de modo que se asigne un número real a cada uno de los resultados posibles del experimento, por lo cual será necesario definir ciertas funciones reales asociadas a una variable aleatoria.
\\\\
Consideremos el caso discreto. Sea $X$ una variable aleatoria discreta que toma los valores $x_0,x_1,x_2,\ldots$ con probabilidades
$$p_0=P(X=x_0)$$
$$p_1=P(X=x_1)$$
$$p_2=P(X=x_2)$$
$$\vdots$$
Esta lista de valores numéricos y sus probabilidades puede ser finita o infinita, pero numerable. La función de probabilidad de $X$ se definiría como aquella función que toma estas probabilidades como valores.
\begin{Def}
    Sea X una variable aleatoria discreta con valores reales $x_1,x_2,\ldots$, $P$ la medida de probabilidad inducida por la variable aleatoria $X$. La función de probabilidad de $X$ es una función  $f_{X}:\R\rightarrow\R$ que se define como
    \begin{eqnarray*}
        f_{X}(x)=
        \begin{cases}
            P(X=x) &\textit{Si }x=x_1,x_2,\ldots\\0 &\textit{en otro caso}\label{def-funcionProbabilidad}
        \end{cases}
    \end{eqnarray*}
\end{Def}
La función de probabilidad es aquella función que indica la probabilidad que tiene la variable aleatoria $X$ de ser $x$.
\begin{Def}
%%%%%% DISTRIBUCIÓN DE POISSON %%%%%%%%%%%%%%%
\label{def-distribuciónProb}
    Al conjunto $\{f(x_i): x_i\in S\}$ o $\{P(X=x_i): x_i\in S\} $se le conoce como distribución de probabilidad.
\end{Def}
Si queremos hallar la probabilidad de algún evento $B\subset\R$ y si la variable aleatoria asociada $X$ es discreta, entonces
$$P(B)=P\big(X\in B)=P\big(\bigcup_{x\in B}(X=x)\big)=\sum_{x\in B}P(X=x)=\sum_{x\in B}f_X(x)$$
De esta manera, la función de probabilidad $f_X$ muestra la forma en la que la probabilidad se distribuye sobre el conjunto discreto $\{x_0,x_1,x_2,\ldots\}$ 
Si $f$ es una función de probabilidad asociada a $X$ se cumple que 
\begin{eqnarray}
    f(x)\geq 0,\quad\sum_{x\in \R}f(x)=1 \label{prop-variableAleatoria-sumaUno}
\end{eqnarray}.
Recíprocamente, toda función cuyo conjunto de puntos de discontinuidad es numerable que cumpla las dos propiedades de (\ref{prop-variableAleatoria-sumaUno}) será llamada también función de probabilidad sin que haya de por medio una variable aleatoria.\\
Algunas distribuciones son muy recurrentes y tienen aplicaciones muy útiles.
\begin{Ejm}
\label{ejm-variableAleatoria-poison-bacteria}
    Supongamos que deseamos  registrar el número de de bacterias por $cm^2$ de cultivo. Para modelar este tipo de situación podemos definir la variable aleatoria $X$ como el número de bacterias que aparecen en $1$ hora de observación en $1 cm^2$ de cultivo. $X$ puede tomar los valores $0,1,2,\ldots$ y en principio no ponemos una cota superior para este número.\\La probabilidad de que existan $n$ bacterias por $cm^2$ de cultivo sería denotada por $P(X=n)$, donde $P$ es la probabilidad inducida por $X$.
\end{Ejm}
Situaciones como la expuesta en el ejemplo  (\ref{ejm-variableAleatoria-poison-bacteria}) son bastantes recurrentes y por ello es necesario definir una distribución que modele estas situaciones.
\begin{Def}(Distribución de Poisson)
    Sea $X$ una variable aleatoria que toma valores $0,1,2,\ldots$ , esta tiene una distribución de Poisson con parámetro $\lambda>0$ y será denotado $X\sim Poisson(\lambda)$ cuando su función de probabilidad es
    $$f(x)=\begin{cases}\frac{e^{-\lambda}\lambda^x}{x!}& \textit{Si }x=0,1,2,\ldots\\
    0 &\textit{en otro caso}
    \end{cases}$$
    El parámetro $\lambda$ se interpreta como el número promedio de ocurrencias del evento por unidad de tiempo o espacio.
\end{Def}
\begin{Ejm}
    Volviendo al caso del ejemplo (\ref{ejm-variableAleatoria-poison-bacteria}), la cual adicionalmente conocemos que la tasa media de bacterias por $cm^2$ es $10$. Si queremos calcular la probabilidad de que existan $8$ bacterias por $cm^2$ usamos la distribución de Poisson, 
    $$f(8)=\frac{e^{-10} 10^{8}}{8!}=  0,112599032. $$ La representación gráfica para este caso,
    \begin{figure}
        \includegraphics[width=15cm]{Cap1-Probabilidad/img/poisson.png}
    \end{figure}
\end{Ejm}
Esta distribución es una de las más importantes de una variable discreta. Sus principales aplicaciones hacen referencia a la modelización de situaciones en las que nos interesa determinar el número de hechos de cierto tipo que se pueden producir en un intervalo de tiempo o de espacio, bajo presupuestos de aleatoriedad y ciertas circunstancias restrictivas.\\
Esta distribución se puede hacer derivar de un proceso experimental de observación en el que tengamos las siguientes características.
\begin{itemize}
    \item El experimento debe ser aleatorio.
    \item Se observa el experimento durante un cierto periodo de tiempo o a lo largo de un espacio de observación.
    \item La probabilidad de que se produzcan un número $n$ de éxitos en un intervalo de amplitud $t$ no depende del origen del intervalo (aunque, sí de su amplitud).
    \item La probabilidad de que ocurra un hecho en un intervalo infinitésimo es prácticamente proporcional a la amplitud del intervalo.
    \item La probabilidad de que se produzcan $2$ o más hechos en un intervalo infinitésimo es un infinitésimo de orden superior a dos.
\end{itemize}
Gracias a estas características podemos identificar cuando nos encontramos en el caso de una distribución de Poisson. \\\\
%%%%%%%%%%%%%%%%%%%%%% DISTRIBUCIÓN BINOMIAL %%%%%%%%%%%%%%%%%%%%%
Otra distribución sumamente útil es la binomial, para cual primera será necesario definir la distribución de Bernoulli.\\ Esta distribución se define como aquel experimento aleatorio con únicamente dos posibles resultados, llamados genéricamente: éxito y fracaso.
Supondremos que las probabilidades de estos resultados son $p$ y $1-p$, respectivamente. Si se define la variable aleatoria $X$ como aquella función que lleva el resultado éxito al número $1$ y el resultado fracaso al número $0$, entonces decimos que $X$ tiene una distribución Bernoulli con parámetro $p\in[0,1]$.\\
Ahora, supongamos que efectuamos una serie de $n$  ensayos independientes Bernoulli en donde la probabilidad de éxito en cada ensayo es $p$. Si denotamos por $E$ el resultado éxito y por $F$ el resultado fracaso, entonces el espacio muestral de este experimento consiste de todas las posibles sucesiones de longitud $n$ de caracteres $E$ y $F$ . Así el espacio muestral consiste de $2^n$ elementos. Si ahora definimos la variable aleatoria $X$ como aquella función que indica el número de éxitos en cada una de estas sucesiones, esto es,
\begin{eqnarray*}
    X(E E\ldots E)=n\\ X(FE\ldots E)=n-1\\ \vdots\\X(F F\ldots F)=0
\end{eqnarray*}
Notamos que $X$  puede tomar los valores $0,1,\ldots,n$  y esta variable aleatoria devolverá el número de éxitos al realizar un número determinado de experimentos. Analicemos esto con un ejemplo.
\begin{Ejm}
    Se realiza un experimento con un tipo de fertilizante orgánico para eliminar el musgo en una plantación. Se encontró una efectividad en los primeros experimentos del $75\%$.
    Si se aplica el mismo fertilizante en $3$ parcelas del mismo tamaño y bajo las mismas condiciones, esto nos dice que tenemos $3$ ensayos independientes Bernoulli con probabilidad de éxito de $p=0.75$  cada una.\\
    Nuestro espacio muestral estaría conformado por $$\Omega= \{E E E,\thinspace F E E,\thinspace E F E,\thinspace E E F,\thinspace F F E,\thinspace F E F,\thinspace E F F,\thinspace F F F\}$$
    La probabilidad de obtener que $2$ parcelas no pierdan su cosecha es preliminarmente,
    \begin{eqnarray}
        \label{eq-ejm-binomial-fertilizante}
    	p\thinspace p (1-p)=(0.75)^2(0.25)
    \end{eqnarray}
    Aquí hemos colocado los $2$ éxitos en los primeros ensayos, cuando ello no ocurrirá necesariamente así. Las diferentes formas en que los $2$ éxitos pueden distribuirse de $3$ formas distintas $(F E E,\thinspace E F E,\thinspace E E F)$, al multiplicarlo con (\ref{eq-ejm-binomial-fertilizante}) obtenemos la probabilidad que deseamos. $$3(0.75)^2(0.25)$$
\end{Ejm}

En general, las diferentes formas en que los $x$ éxitos pueden distribuirse en los $n$ ensayos está dada por el coeficiente binomial $(n\choose x)$, al hacer la multiplicación de este coeficiente binomial con el término $p^x(1-p)^{n-x}$ se obtiene la expresión de la función de probabilidad para esta distribución.
\begin{Def}
    Una variable aleatoria $X$ tiene una distribución binomial con parámetros $n$ y $p$ y se denota por $X\sim bin(n,p)$ si tiene por función de probabilidad
    $$f(x)=
    \begin{cases}
        {n \choose x} p^x(1-p)^{n-x}& x=0,1,\ldots,n\\0 & \textit{en otro caso}
    \end{cases}$$
\end{Def}
Un experimento se puede modelar con una distribución binomial si cumple que:
\begin{itemize}
    \item Sólo hay dos posibles sucesos resultantes del experimento:
    (éxito y fracaso).
    \item Las probabilidades de cada suceso son las mismas en cualquier realización del experimento ( $p$ y $1-p$ , respectivamente).
    \item Toda realización del experimento es independiente del resto.
\end{itemize}
Gracias a estas características podemos identificar cuando un experimento modela una distribución binomial.\\\\
%%%%%%%%%%%%%%%% DISTRIBUCIÓN EXPONENCIAL%%%%%%%%%%%%%%%%%%%%%%%%%%%%%%
\begin{Def}
    Decimos que una variable aleatoria continua $X$ tiene distribución exponencial con parámetro $\lambda>0$, cuando su función de densidad es
    $$\begin{cases}
        \lambda e^{-\lambda x}, & \mbox{si $x>0$}\\
        0, & \mbox{en otro caso}
    \end{cases}
    $$
\end{Def}
Se trata pues de una variable aleatoria continua con con-
junto de valores el intervalo $[0,\infty)$. Esta distribución se usa para modelar tiempos de espera para la ocurrencia de un cierto evento.\\
A pesar de la sencillez analítica de sus funciones de definición, la distribución exponencial tiene una gran utilidad práctica ya que podemos considerarla como un modelo adecuado para la distribución de probabilidad del tiempo de espera entre dos hechos que sigan un proceso de Poisson.\\
Tiene una gran utilidad en los siguientes casos:
\begin{itemize}
    \item Distribución del tiempo de espera entre sucesos de un proceso de Poisson.
    \item Distribución del tiempo que transcurre hasta que se produce un fallo, si se cumple la condición que la probabilidad de producirse un fallo en un instante no depende del tiempo transcurrido.
\end{itemize}
%%%%%%%%%%%%%%%%%%%%
Existen muchas más distribuciones recurrentes para variables aleatorias discretas, pero las mencionadas serán de nuestro principal interés.