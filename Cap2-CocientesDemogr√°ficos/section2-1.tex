\section{Tipos de cocientes más relevantes}
En análisis demográfico existen cuatro tipos de tales divisiones o cocientes, distinguidos en función del tipo de datos que constituyen respectivamente el numerador y el denominador.
\begin{itemize}
    \item Cuando ambos números son el mismo tipo de magnitud, bien flujos, bien stocks.
    \begin{enumerate}
        \item Proporción. Cociente que resulta de dividir un subconjunto por el conjunto total en el que está incluido.\\
        Por ejemplo, las mujeres de una población respecto a la población total.
        \item Razón. Cociente que resulta de dividir dos conjuntos o subconjuntos distintos que no tienen elementos comunes. Por ejemplo, los hombres de una población respecto a las mujeres de esa misma población, la llamada "razón de masculinidad".
    \end{enumerate}
    \item Cuando el numerador es un flujo de acontecimientos y sólo el denominador es un stock poblacional.
    \begin{enumerate}
        \item Tasa. Cociente que resulta de dividir un número de acontecimientos sucedidos durante un periodo de tiempo (un flujo) por la población media existente durante ese periodo.\\Por ejemplo la tasa de mortalidad es el número de defunciones durante un periodo de tiempo, dividido por la población media de ese mismo periodo. \\ Lo que diferencia las tasas de las proporciones es que en el numerador se sitúan flujos, es decir, acontecimientos registrados durante cierto periodo, mientras que el denominador corresponde a un stock y, por tanto, se refiere a un instante.
        \item Probabilidad. Cociente entre los acontecimientos experimentados por una población durante un periodo de tiempo y la población inicial de dicho periodo, susceptible de experimentar tales acontecimientos.Adoptan la forma de una fracción en que la población afectada ocupa el numerador y en el denominador (abajo) se sitúa la población que podría en principio haber estado afectada.\\Por ejemplo, la probabilidad de morir entre los 20 y los 22 años para la generación 1970 se calcula dividiendo las defunciones de miembros de esta generación, en ese intervalo de edades, por el número de sus integrantes que se encontraban con vida al cumplir los 20 años exactos.
    \end{enumerate}
\end{itemize}
El análisis demográfico permiten mejorar la toma de decisiones y hacer pronósticos sobre determinadas cuestiones, por ejemplo, en torno a la salud, a determinadas acciones a tomar frente un catástrofe o a las políticas económicas.
