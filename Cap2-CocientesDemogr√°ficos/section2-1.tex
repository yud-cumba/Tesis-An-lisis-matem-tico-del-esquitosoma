\section{Tipos de cocientes más relevantes}
En análisis demográfico existen cuatro tipos de tales divisiones o cocientes, distinguidos en función del tipo de datos que constituyen respectivamente el numerador y el denominador.
\begin{itemize}
    \item Cuando ambos números son el mismo tipo de magnitud, bien flujos, bien stocks.
    \begin{enumerate}
        \item Proporción. Cociente que resulta de dividir un subconjunto por el conjunto total en el que está incluido.\\
        Por ejemplo, las mujeres de una población respecto a la población total.
        \item Razón. Cociente que resulta de dividir dos conjuntos o subconjuntos distintos que no tienen elementos comunes. Por ejemplo, los hombres de una población respecto a las mujeres de esa misma población, la llamada "razón de masculinidad".
    \end{enumerate}
    \item Cuando el numerador es un flujo de acontecimientos y sólo el denominador es un stock poblacional.
    \begin{enumerate}
        \item Tasa. Cociente que resulta de dividir un número de acontecimientos sucedidos durante un periodo de tiempo (un flujo) por la población media existente durante ese periodo.\\Por ejemplo la tasa de mortalidad es el número de defunciones durante un periodo de tiempo, dividido por la población media de ese mismo periodo.
        \item Probabilidad. Cociente entre los acontecimientos experimentados por una población durante un periodo de tiempo y la población inicial de dicho periodo, susceptible de experimentar tales acontecimientos.\\Por ejemplo, la probabilidad de morir de $x$ personas que cumplen 20 años durante el siguiente año de su vida.
    \end{enumerate}
\end{itemize}
Las tasas pueden interpretarse como la frecuencia relativa con que se producen ciertos acontecimientos en relación a la población media existente durante el tiempo en que se han registrado tales acontecimientos. Las más conocidas son las tasas de mortalidad y de natalidad.\\Lo que diferencia las tasas de las proporciones es que en el numerador se sitúan flujos, es decir, acontecimientos registrados durante cierto periodo, mientras que el denominador corresponde a un stock y, por tanto, se refiere a un instante.
\\\\
Las probabilidades pueden interpretarse como la relación numérica entre los sujetos susceptibles de experimentar un determinado fenómeno (población) y los fenómenos efectivamente acontecidos después, pasado cierto tiempo. Adoptan la forma de una fracción en que la población afectada ocupa el numerador (arriba), y en el denominador (abajo) se sitúa la población que podría en principio haber estado afectada. El resultado puede transformarse en un porcentaje o un tanto por mil. Por ejemplo, la probabilidad de morir entre los $20$ y los $22$ años para la generación 1970 se calcula dividiendo las defunciones de miembros de esta generación, en ese intervalo de edades, por el número de sus integrantes que se encontraban con vida al cumplir los $20$ años exactos. Las probabilidades son el elemento básico para la construcción de las tablas de eliminación, como las de mortalidad, nupcialidad, fecundidad, etc.\\\\
El análisis demográfico permiten mejorar la toma de decisiones y hacer pronósticos sobre determinadas cuestiones, por ejemplo, en torno a la salud, a determinadas acciones a tomar frente un catástrofe o a las políticas económicas.
