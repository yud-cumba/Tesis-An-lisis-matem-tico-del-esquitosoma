\section{Flujo y stock}
\begin{Def}
    Una población es un conjunto renovable de individuos que cumplen determinada condición, sean humanos, animales o cosas y no son un conjunto estático, sino que están sometidas a un proceso continuo de cambio, por salidas y entradas de individuos en dicha población.
\end{Def}
Los indicadores estadísticos utilizados en demografía casi nunca son demasiado complicados. De hecho, los más frecuentes son las que tienen relación entre solo dos magnitudes. Ese caso concreto de indicador en el que dos magnitudes se ponen en relación, se conoce como “cociente”, y hay tipos muy diversos en función de la naturaleza de los datos que usemos. Para entender lo que sigue es importante primero saber distinguir entre lo que significa “flujo” y “stock”.
\begin{Def}
    Los stocks son el número de individuos en un punto exacto en el tiempo.\\
    Tienen dimensión temporal instantánea. Se refieren a las existencias en un determinado momento.\\
    Los datos de stocks poblacionales pueden tener como fuente los censos, padrones, estimaciones de población o encuestas.\\
    Un ejemplo de stock sería el número de personas vivas el 25 de Diciembre del 2019.
\end{Def}
\begin{Def}
    Los flujos son los acontecimientos o fenómenos (como los nacimientos o las defunciones) en determinado espacio temporal continuo.\\Los datos de flujo provienen de “registros” de acontecimientos, como el Registro Civil de nacimientos, defunciones o matrimonios.\\
    Un ejemplo de flujo sería el número de nacidos vivos durante el año 2019.
\end{Def}