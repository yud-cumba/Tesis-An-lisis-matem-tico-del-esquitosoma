\section{Tipos de cocientes más relevantes}
"En el análisis demográfico, existen cuatro tipos de tales divisiones o cocientes, que se distinguen por el tipo de datos que forman el numerador y el denominador respectivamente".
\begin{itemize}
    \item Cuando ambos números (tanto el denominador como el numerador) son el mismo tipo de magnitud pueden ser los llamados flujos o bien stocks.
    \begin{enumerate}
        \item Proporción. Cociente que resulta de dividir un subconjunto por el conjunto total en el que está incluido.\\
        Por ejemplo, las mujeres de una población respecto a la población total.
        \item Razón. "Cociente que resulta de dividir dos conjuntos o subconjuntos distintos que no tienen elementos comunes. Por ejemplo, los hombres de una población respecto a las mujeres de esa misma población, la llamada razón de masculinidad".
    \end{enumerate}
    \item "Cuando el numerador es un flujo de acontecimientos y sólo el denominador es un stock poblacional".
    \begin{enumerate}
        \item Tasa. "Es el cociente es el resultado de dividir el número de eventos (flujos) que ocurrieron durante un período entre la población promedio que existe durante ese período. \\ Por ejemplo, la mortalidad es el número de personas que fallecieron en un período determinado dividido por la población promedio en el mismo período. \\ La diferencia entre razones y proporciones es que el flujo se coloca en el numerador. Es decir, el evento se registra por un período de tiempo, pero el denominador corresponde al stock, por lo que se menciona de una vez".
        \item Probabilidad. "Cociente entre un evento experimentado por una población durante un período de tiempo y la primera población de ese período vulnerable a tal evento. Se presentan en forma de fracciones, donde la población afectada ocupa el numerador y el denominador (abajo) es, en principio, la población que puede haber sido afectada. \\ Por ejemplo, la probabilidad de morir entre los 20 y los 22 años en la generación de 1970 se calcula dividiendo el número de muertes de miembros de esta generación en este rango de edad por el número de miembros restantes. 20 años de edad".
    \end{enumerate}
\end{itemize}
El análisis demográfico permiten mejorar la toma de decisiones y hacer pronósticos sobre determinadas cuestiones, por ejemplo, en torno a la salud, a determinadas acciones a tomar frente un catástrofe o a las políticas económicas.
