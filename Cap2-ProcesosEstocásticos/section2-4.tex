\section{Teorema de Daniell-Kolgomorov}
    Revisamos la construcción general de un proceso estocástico.
    Para fines de modelización nos gustaría considerar las familias de variable aleatoria $\{X_t\}_{t\in I}$ indexada por un conjunto general $I$. Cuando $I$ es discreto y finito, la familia vendría a ser un vector aleatorio con valores en el espacio producto medible $(E,\mathscr{E}^I)$ donde  $(E, \mathscr{E})$  es el espacio de medida donde se define cada $X_t$ , es decir, $X_t:(\Omega, \mathscr{F}) \rightarrow (E, \mathscr{E})$ para todo $t\in I$.
     Tomando $\pi_t: (E^I,  \mathscr{E}^I) \rightarrow (E,  \mathscr{E})$ como la función proyección en el factor $t$, se puede definir simplemente $X:(\Omega, \mathscr{F})\rightarrow (E^I,  \mathscr{E}^I)$ con $\pi_t X = X_t$.
      De esta forma hemos reemplazado la familia de variables aleatorias  que  cada uno toma valores en $E$ con una sola variable aleatoria que $X$ toma valores en $(E^I, c )$ donde $\mathscr{E}^I$  es el  $\sigma$-álgebra producto.
    $\mathscr{E}^I$ es el sigma álgebra generado por conjuntos de la forma $\Pi_{i\in I}{A_i}\subset E^I$, donde $A_i\in\mathscr{E}, \forall i\in I$
    Si $I$ no es finito o incluso no es contable este procedimiento tiene que ser aclarado.
    Buscamos una definición para $\mathscr{E}^I$ que mantenga que para todo $t\in I$, la función proyección $\pi_i: (E^I,\mathscr{E}^I)\rightarrow (E,\mathscr{E})$ permanezca medible.
    \\La familia de conjuntos de la forma $$\pi^{-1}_J(A), \quad J\subseteq I, \textit{J finito, }A\in\mathscr{E}^J,$$  donde $\pi_J :E^I\rightarrow E^J$	 es la proyección en las coordenadas en $J$, constituye un álgebra y son llamados conjuntos cilindros. 
    \\La familia de conjuntos cilindros de la forma  $$\pi^{-1}_J\bigg(\prod_{t\in J}A_t\bigg), \quad J\subseteq I, \textit{J finito, }A_t\in\mathscr{E}, \textit{ para todo }t\in J,$$
    forman un $\pi$-sistema dentro del álgebra de los conjuntos cilindros.
    Consideramos $\mathscr{E}^I$ como el $\sigma-$álgebra generado por la familia de conjuntos cilindros.
    Esta definición está concebido para lograr la siguiente equivalencia.
    \begin{Lem}
      Una función $X:(\Omega,\mathscr{F})\rightarrow (E^I,\mathscr{E}^I)$ es medible si y solo si $\pi_t X:(\Omega,\mathscr{F})\rightarrow (E,\mathscr{E})$ para todo $t\in I$
    \end{Lem}
    Dado un proceso $X:(\Omega,\mathscr{F})\rightarrow (E^I,\mathscr{E}^I)$ llamamos a la medida $\mu X$ en $(E^I,\mathscr{E}^I)$ dada por $\mu X(A)= P(X\in A)$ para todo $A\in I$. Además, como ya se vio, en el espacio $(E^I,\mathscr{E}^I)$ siempre podemos realizar el proceso estocástico $(X_t: \Omega \rightarrow E)$ $t\in I$ tomando $X_t(\omega)=\omega_t$ para que $X(\omega)=\omega$. Esto se llama el proceso canónico.
    \\
  El siguiente teorema, debido a Kolmogorov (e independientemente a Daniell) establece la existencia de una probabilidad en $(E^I,\mathscr{E}^I)$ gracias a la existencia de una familia de probabilidades que cumplen ciertas condiciones.
\begin{Def}
    Decimos que una familia de probabilidades $\{\mu_J\}_{J\subset I}$, donde $J$ es finito, es consistente si para cualquier $J'\subset J$ y con $\pi_{J, J}:E^J \rightarrow E^{J'}$ la proyección canónica de $E^J$ a $E^{J'}$, tenemos
     que $$\mu_J \circ \pi_{J,J'}^{-1}= \mu_j$$
\end{Def}
\begin{Def}
    Sea $(X,\mathscr{T})$ un espacio topológico y $\mathscr{F}$ un $\sigma$-álgebra sobre $X$ y $\mu$ la medida sobre el espacio medible $(X, \mathscr{F})$.Un subconjunto medible $A$ de $X$ se dice que es normal interior si
    $$\{\mu(A) = \sup \{\mu(F) |\thinspace F \subseteq A, F \textit{ compacto y medible} \}$$
    Una medida se llama regular de interior si cada conjunto medible es regular interior.
\end{Def}
\begin{Teo}
    Asumamos que $\{\mu_J\}_{J\subset I}$, donde $J$ es finito es una familia de probabilidades que son regular de interior. Entonces existe una única medida $\mu$ en $(E^I, \mathscr{E}^I)$ tal que $\mu\circ\pi_J^{-1}= \mu_J$, para todo $J\subset I$
\end{Teo}