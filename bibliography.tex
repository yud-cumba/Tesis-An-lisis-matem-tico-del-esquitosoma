\clearpage
\addcontentsline{toc}{chapter}{References}
\begin{thebibliography}{25}
%https://es.wikipedia.org/wiki/Variable_aleatoria
\bibitem{(JordanyWebbe} (Jordan y Webbe [5], p.
56)
\bibitem{esquitomasis}Nasell,I. , Hirsch, W.(1973)\emph{ Communications on Pure and Applied Mathematics Volume 26 issue 4[doi 10.10022Fcpa.3160260402]The transmission dynamics of schistosomiasis}
\bibitem{Feller}Feller, W. (1968)\emph{ An Introduction to Probability Theory and Its Applications, Vol. 1, 3rd Edition}
\bibitem{Rincon1} Rincón, L. (2014) \emph{Introducción a la probabilidad}
\bibitem{Rincon2} Rincón, L. (2010) \emph{Curso intermedio de probabilidad}
\bibitem{Rincon3} Rincón, L. (2010) \emph{ Introducción a los procesos estocásticos}
\bibitem{LindaAllen} Allen, L. (2010)\emph{ An Introduction to Stochastic Processes with Applications to Biology, Second Edition-Chapman and Hall-CRC}
\bibitem{cimatModelaje} García N.(2007), \emph{Teoría de la medida y la probabilidad}
\bibitem{libroluyo} Anderson, W.(2011)\emph{Métodos cuantitativos para los negocios, 11a Edition}
\bibitem{curso_medida_rotger} Metzger, R.(2008) \emph{Curso Básico de Teoria de la Medida}
\bibitem{smsm} Zachmanoglou, E. (1976) \emph{Introduction to differential partial equations with aplications}
\bibitem{d} Vasy, A. (2015) \emph{Partial differential equations: an accessible route through theory and applications}
\bibitem{intro-probabilidad} García, M. (2003) \emph{Introducción a la teoría de la probabilidad}
\end{thebibliography}

\bibliography{tesis}